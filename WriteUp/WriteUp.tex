\documentclass[12pt,preprint, authoryear]{elsarticle}

\usepackage{lmodern}
%%%% My spacing
\usepackage{setspace}
\setstretch{1.2}
\DeclareMathSizes{12}{14}{10}{10}

% Wrap around which gives all figures included the [H] command, or places it "here". This can be tedious to code in Rmarkdown.
\usepackage{float}
\let\origfigure\figure
\let\endorigfigure\endfigure
\renewenvironment{figure}[1][2] {
    \expandafter\origfigure\expandafter[H]
} {
    \endorigfigure
}

\let\origtable\table
\let\endorigtable\endtable
\renewenvironment{table}[1][2] {
    \expandafter\origtable\expandafter[H]
} {
    \endorigtable
}


\usepackage{ifxetex,ifluatex}
\usepackage{fixltx2e} % provides \textsubscript
\ifnum 0\ifxetex 1\fi\ifluatex 1\fi=0 % if pdftex
  \usepackage[T1]{fontenc}
  \usepackage[utf8]{inputenc}
\else % if luatex or xelatex
  \ifxetex
    \usepackage{mathspec}
    \usepackage{xltxtra,xunicode}
  \else
    \usepackage{fontspec}
  \fi
  \defaultfontfeatures{Mapping=tex-text,Scale=MatchLowercase}
  \newcommand{\euro}{€}
\fi

\usepackage{amssymb, amsmath, amsthm, amsfonts}

\def\bibsection{\section*{References}} %%% Make "References" appear before bibliography


\usepackage[round]{natbib}

\usepackage{longtable}
\usepackage[margin=2.3cm,bottom=2cm,top=2.5cm, includefoot]{geometry}
\usepackage{fancyhdr}
\usepackage[bottom, hang, flushmargin]{footmisc}
\usepackage{graphicx}
\numberwithin{equation}{section}
\numberwithin{figure}{section}
\numberwithin{table}{section}
\setlength{\parindent}{0cm}
\setlength{\parskip}{1.3ex plus 0.5ex minus 0.3ex}
\usepackage{textcomp}
\renewcommand{\headrulewidth}{0.2pt}
\renewcommand{\footrulewidth}{0.3pt}

\usepackage{array}
\newcolumntype{x}[1]{>{\centering\arraybackslash\hspace{0pt}}p{#1}}

%%%%  Remove the "preprint submitted to" part. Don't worry about this either, it just looks better without it:
\makeatletter
\def\ps@pprintTitle{%
  \let\@oddhead\@empty
  \let\@evenhead\@empty
  \let\@oddfoot\@empty
  \let\@evenfoot\@oddfoot
}
\makeatother

 \def\tightlist{} % This allows for subbullets!

\usepackage{hyperref}
\hypersetup{breaklinks=true,
            bookmarks=true,
            colorlinks=true,
            citecolor=blue,
            urlcolor=blue,
            linkcolor=blue,
            pdfborder={0 0 0}}


% The following packages allow huxtable to work:
\usepackage{siunitx}
\usepackage{multirow}
\usepackage{hhline}
\usepackage{calc}
\usepackage{tabularx}
\usepackage{booktabs}
\usepackage{caption}


\newenvironment{columns}[1][]{}{}

\newenvironment{column}[1]{\begin{minipage}{#1}\ignorespaces}{%
\end{minipage}
\ifhmode\unskip\fi
\aftergroup\useignorespacesandallpars}

\def\useignorespacesandallpars#1\ignorespaces\fi{%
#1\fi\ignorespacesandallpars}

\makeatletter
\def\ignorespacesandallpars{%
  \@ifnextchar\par
    {\expandafter\ignorespacesandallpars\@gobble}%
    {}%
}
\makeatother

\newenvironment{CSLReferences}[2]{%
}

\urlstyle{same}  % don't use monospace font for urls
\setlength{\parindent}{0pt}
\setlength{\parskip}{6pt plus 2pt minus 1pt}
\setlength{\emergencystretch}{3em}  % prevent overfull lines
\setcounter{secnumdepth}{5}

%%% Use protect on footnotes to avoid problems with footnotes in titles
\let\rmarkdownfootnote\footnote%
\def\footnote{\protect\rmarkdownfootnote}
\IfFileExists{upquote.sty}{\usepackage{upquote}}{}

%%% Include extra packages specified by user
\usepackage{booktabs}
\usepackage{longtable}
\usepackage{array}
\usepackage{multirow}
\usepackage{wrapfig}
\usepackage{float}
\usepackage{colortbl}
\usepackage{pdflscape}
\usepackage{tabu}
\usepackage{threeparttable}
\usepackage{threeparttablex}
\usepackage[normalem]{ulem}
\usepackage{makecell}
\usepackage{xcolor}

%%% Hard setting column skips for reports - this ensures greater consistency and control over the length settings in the document.
%% page layout
%% paragraphs
\setlength{\baselineskip}{12pt plus 0pt minus 0pt}
\setlength{\parskip}{12pt plus 0pt minus 0pt}
\setlength{\parindent}{0pt plus 0pt minus 0pt}
%% floats
\setlength{\floatsep}{12pt plus 0 pt minus 0pt}
\setlength{\textfloatsep}{20pt plus 0pt minus 0pt}
\setlength{\intextsep}{14pt plus 0pt minus 0pt}
\setlength{\dbltextfloatsep}{20pt plus 0pt minus 0pt}
\setlength{\dblfloatsep}{14pt plus 0pt minus 0pt}
%% maths
\setlength{\abovedisplayskip}{12pt plus 0pt minus 0pt}
\setlength{\belowdisplayskip}{12pt plus 0pt minus 0pt}
%% lists
\setlength{\topsep}{10pt plus 0pt minus 0pt}
\setlength{\partopsep}{3pt plus 0pt minus 0pt}
\setlength{\itemsep}{5pt plus 0pt minus 0pt}
\setlength{\labelsep}{8mm plus 0mm minus 0mm}
\setlength{\parsep}{\the\parskip}
\setlength{\listparindent}{\the\parindent}
%% verbatim
\setlength{\fboxsep}{5pt plus 0pt minus 0pt}



\begin{document}



\begin{frontmatter}  %

\title{Time-varying Correlations Between Sectors of the All Share Index}

% Set to FALSE if wanting to remove title (for submission)




\author[Add1]{Wesley Williams}
\ead{21691126@sun.ac.za}





\address[Add1]{Final Project for Financial Econometrics 871 at
Stellenosch University, South Africa}



\vspace{1cm}





\vspace{0.5cm}

\end{frontmatter}

\setcounter{footnote}{0}



%________________________
% Header and Footers
%%%%%%%%%%%%%%%%%%%%%%%%%%%%%%%%%
\pagestyle{fancy}
\chead{}
\rhead{}
\lfoot{}
\rfoot{\footnotesize Page \thepage}
\lhead{}
%\rfoot{\footnotesize Page \thepage } % "e.g. Page 2"
\cfoot{}

%\setlength\headheight{30pt}
%%%%%%%%%%%%%%%%%%%%%%%%%%%%%%%%%
%________________________

\headsep 35pt % So that header does not go over title




\hypertarget{introduction}{%
\section{\texorpdfstring{Introduction
\label{Introduction}}{Introduction }}\label{introduction}}

Diversification is the cornerstone of modern day finance, thus
understanding the co-movements of assets is essential for any
diversification or hedging strategy. Katzke
(\protect\hyperlink{ref-katzke2013south}{2013}) highlights that
investors have a ``home-bias'', where local assets are favoured over
international assets. This can lead to risks of underdiversification
when there is homogoneity in return movements between local sectors and
assets. This project, therefore, aims to identify and investigate the
co-movements of three of the largest sectors in South Africa.

The use of Mulitvariate Generalized Autoregressive Conditional
Heteroskedasticity (MV-GARCH) to obtain dynamic conditional correlations
has become more prevalent in finance and portfolio theory literature. It
has been used for many different purposes. Ho \& Tsui
(\protect\hyperlink{ref-ho2004analysis}{2004}) \& Katzke
(\protect\hyperlink{ref-katzke2013south}{2013}) employ MV-GARCH models
to investigate diversification opportunities and threats through
inter-sector correlations. Fakhfekh, Jeribi, Ghorbel \& Hachicha
(\protect\hyperlink{ref-fakhfekh2021hedging}{2021}) \& Ali, Raza, Vo \&
Le (\protect\hyperlink{ref-ali2022modelling}{2022}) instead look at
hedging strategies utilising a vast array of different financial assets
and instruments.

In this paper I extract three sectors' returns from the All Share J203
in order to derive the time-varying correlations between the three
sectors and the United States Dollar to South African Rand exchange
rate. The rest of the paper is structured as follows, section two
describes the data, section three desribes the methods and statistical
tests used, section four presents the final results and lastly, section
five concludes.

\hypertarget{data}{%
\section{Data}\label{data}}

The data used to get the sector returns is the All Share (ALSI) J203
index. This index represents 99\% of the full market cap value of all
eligible securities listed on the Main Board of the Johannesburg Stock
Exchange (JSE) (\protect\hyperlink{ref-jse_website}{JSE, 2024}). Sector
returns are then created for the financial, industrial and resource
sectors. All stocks categorised in each of these sectors is reweighted
in order to create a index for each sector. The cumulative returns are
presented in figure \ref{Figure1}.

Other data that was used is the United States Dollar (USD) to South
African Rand (ZAR) exchange rate as the depreciation or appreciation of
the domestic currency compared to the Dollar is an indicator of changes
in the global economy. Lastly the historical repurchase rate (REPO) was
obtained from the South African Reserve Bank (SARB).

\begin{figure}[H]

{\centering \includegraphics{WriteUp_files/figure-latex/Figure1-1} 

}

\caption{Cumulative Returns of ALSI \label{Figure1}}\label{fig:Figure1}
\end{figure}

\hypertarget{methodology}{%
\section{Methodology}\label{methodology}}

I begin by conducting multiple tests for autoregressive conditional
heteroskedasticity (ARCH) effects. I do this through graphing the
returns, absolute returns and squared returns which can be found in
figures \ref{FigureA1}, \ref{FigureA2} and \ref{FigureA3}. These figures
highlights that the return series of all three sectors show strong first
order persistence and potential periods of second order persistence.
Lastly, it appears that the series has a long memory in the second order
process. To check for this I conduct more formal tests for ARCH effects.

\hypertarget{tests-for-autoregressive-conditional-heteroskedasticity-arch-effects.}{%
\subsection{Tests for Autoregressive Conditional Heteroskedasticity
(ARCH)
effects.}\label{tests-for-autoregressive-conditional-heteroskedasticity-arch-effects.}}

The first formal test that I conduct is the plotting of the
autocorrelation functions of the returns, absolute returns and squared
returns for all sectors. Figures \ref{FigureA4}, \ref{FigureA5} and
\ref{FigureA6} can be found in the appendix. These figures provide more
evidence of conditional heteroskedasticity and long memory for all of
the sectors. The last test is a formal Box-Ljung test where the null
hypothesis is that there are no ARCH effects. Table 3.1 presents the
results of the test and it is clear that the null hypothesis of no ARCH
effects can be rejected at the 1\% confidence level. This means that the
condictional heteroskedasticity needs to be controlled for.

\begin{table}

\caption{\label{tab:unnamed-chunk-5}Ljung-Box Test Results}
\centering
\begin{tabular}[t]{l|r|r|r}
\hline
  & TestStatistic & PValue & Lag\\
\hline
Financials & 2667.7971 & 0 & 12\\
\hline
Resources & 1431.3349 & 0 & 12\\
\hline
Industrials & 948.8227 & 0 & 12\\
\hline
\end{tabular}
\end{table}

\hypertarget{univariate-model-selection}{%
\subsection{Univariate Model
Selection}\label{univariate-model-selection}}

A univariate GARCH specification is necassary for the multivariate
specification. There are a vast number of extentions to the original
GARCH formulation, therefore to obtain the model that best fits the data
I employ multiple information criteria to assess the potential model
fits in the univariate case. Table 3.2 presents the results of the
information criteria and the best model for each sector is different but
on average the best fitting model across all sectors is eGARCH so that
is the specification that will be used to estimate the dynamic
conditional correlations.

\begin{table}

\caption{\label{tab:unnamed-chunk-6}GARCH Model Comparison Results}
\centering
\begin{tabular}[t]{l|l|r|r|r|r}
\hline
  & Sector & sGARCH & gjrGARCH & eGARCH & apARCH\\
\hline
Akaike & Financials & -5.872282 & -5.881912 & -5.881702 & -5.884481\\
\hline
Bayes & Financials & -5.861235 & -5.868656 & -5.868446 & -5.869016\\
\hline
Shibata & Financials & -5.872289 & -5.881922 & -5.881712 & -5.884495\\
\hline
Hannan-Quinn & Financials & -5.868285 & -5.877115 & -5.876905 & -5.878885\\
\hline
Akaike1 & Resources & -5.466560 & -5.479735 & -5.475142 & -5.479025\\
\hline
Bayes1 & Resources & -5.455513 & -5.466479 & -5.461886 & -5.463559\\
\hline
Shibata1 & Resources & -5.466567 & -5.479745 & -5.475153 & -5.479038\\
\hline
Hannan-Quinn1 & Resources & -5.462562 & -5.474938 & -5.470345 & -5.473428\\
\hline
Akaike2 & Industrials & -6.246073 & -6.267709 & -6.270655 & -6.261725\\
\hline
Bayes2 & Industrials & -6.235026 & -6.254453 & -6.257399 & -6.246260\\
\hline
Shibata2 & Industrials & -6.246080 & -6.267719 & -6.270665 & -6.261739\\
\hline
Hannan-Quinn2 & Industrials & -6.242075 & -6.262912 & -6.265858 & -6.256129\\
\hline
\end{tabular}
\end{table}

\hypertarget{multivariate-models}{%
\subsection{Multivariate Models}\label{multivariate-models}}

Now that I have the univariate model specifications I can now fit the
multivariate model. I fit three different multivariate GARCH models:
DCC, aDCC and GO-GARCH. This follows the literature as the comparison
between these two or three models is common Ali \emph{et al.}
(\protect\hyperlink{ref-ali2022modelling}{2022}). After fitting the DCC
model I extract the model diagnostics that present multiple Portmanteau
and rank based tests that assess whether there is serial correlation.
Tsay (\protect\hyperlink{ref-tsay2013}{2013}) provides an in depth
discussion about these tests. Th null hypothesis for all the tests are
that there is no serial correlation and hence no conditional
heteroskedasticity. Tsay (\protect\hyperlink{ref-tsay2013}{2013: 403})
notes that \(Q_k (m)\) statistic works well when the distribution of
innovations are normal but struggles when fatter tails are present . The
robust statistic employs 5\% trimming as a means to get a more robust
statistic. The results presented in table \ref{Port} show that for all
tests except for the robust version (\(Q_r^{k} (m)\)) rejects the null
hypothesis of no conditional heteroskedasticity whereas the robust
version fails to reject.

\begin{table}[htbp]
\centering
\caption{Portmanteau tests}
\label{Port}
\begin{tabular}{|c|c|c|c|}
\hline
$Q(m)$ & $Rank-based \ test$ & $Q_k(m)$ & $Q_r^{k} (m)$ \\
\hline
\hline
51.32 & 18.63 & 255.96 & 173.18 \\
(0.000) & (0.045) & (0.000) & (0.225) \\
\hline
\multicolumn{4}{|l|}{Note: P-values given in brackets.} \\
\hline
\end{tabular}
\end{table}

\hypertarget{results}{%
\section{Results}\label{results}}

Figures \ref{Figure2}, \ref{Figure3} and \ref{Figure4} present the time
varying conditional correlations between the three sectors of interest
as well as the USD/ZAR exchange rate. Figure \ref{Figure2} presents the
standard dynamic conditional correlations (DCC). \ref{Figure3} presents
the assymetric dynamic conditional correlations (aDCC), which accounts
for the assymetry of positive and negative shocks. Lastly, figure
\ref{Figure4} presents the time varying correlations from a Generalized
Orthogonal GARCH (GO-GARCH) specification. I also employed
stratification methods to get the most and least volatile periods of the
REPO rate.

Across all of the specifications it is clear to see that during high
periods of volatitlity in the REPO rate the correlations between the
sectors is also more volatile. All specifications also highlight that
there is large heterogeneity in the correlations over time. This serves
as another motivating factor to not use time-invariant estimations of
the correlations. The inter-sector correlations tend to move around 0.5
for both the DCC and aDCC with fluctuations within a band of 0.8 and 0.
The correlation between the financial and industrial sectors is the
highest followed by resources and industrials and then resources and
financials. The correlations between the sectors and the exchange rate
clearly show the financial sector is consistently negatively correlated
with the exchange rate. For the other two sectors a positive correlation
tends to be present during low volatility periods of the REPO and
negative during high volatility periods.

There appears to be little difference between the DCC and aDCC
specifications. When looking at GO-GARCH however, the correlations
instead have much smaller fluctuations. Boswijk \& Weide
(\protect\hyperlink{ref-boswijk2006wake}{2006: 21}) state that the
smaller fluctuations can be seen as positive or a negative depending on
the case. In this case it helps to provide a more consistent measure.
The most important finding is that there does not seem to be a trend in
any of the correlations.

\begin{figure}[H]

{\centering \includegraphics{WriteUp_files/figure-latex/Figure2-1} 

}

\caption{DCC  \label{Figure2}}\label{fig:Figure2}
\end{figure}

\begin{figure}[H]

{\centering \includegraphics{WriteUp_files/figure-latex/Figure3-1} 

}

\caption{aDCC  \label{Figure3}}\label{fig:Figure3}
\end{figure}

\begin{figure}[H]

{\centering \includegraphics{WriteUp_files/figure-latex/Figure4-1} 

}

\caption{GO-GARCH  \label{Figure4}}\label{fig:Figure4}
\end{figure}

\hypertarget{conclusion}{%
\section{Conclusion}\label{conclusion}}

This paper set out to estimate time-varying correlations of three
sectors of the All Share index J203 as well as the USD/ZAR exchange
rate. This was done by first fitting the multiple univariate GARCH
models before evaluating the fit with multiple infomation criteria. Once
the best univariate fit was selected it was used to fit the multivariate
GARCH. The results show that the use of time-varying correlations are
justified and if time-invariant correlations are used it will lead to an
inefficiently diversified portfolio. Secondly, I found that the
financial sector is the most suseptible to Rand depreciation while the
other sectors correlations fluctuate around 0.

\newpage

\hypertarget{references}{%
\section*{References}\label{references}}
\addcontentsline{toc}{section}{References}

\hypertarget{refs}{}
\begin{CSLReferences}{1}{0}
\leavevmode\vadjust pre{\hypertarget{ref-ali2022modelling}{}}%
Ali, S., Raza, N., Vo, X.V. \& Le, V. 2022. Modelling the joint dynamics
of financial assets using MGARCH family models: Insights into hedging
and diversification strategies. \emph{Resources Policy}. 78:102861.

\leavevmode\vadjust pre{\hypertarget{ref-boswijk2006wake}{}}%
Boswijk, H.P. \& Weide, R. van der. 2006. \emph{Wake me up before you
GO-GARCH}. Tinbergen Institute Discussion Paper.

\leavevmode\vadjust pre{\hypertarget{ref-fakhfekh2021hedging}{}}%
Fakhfekh, M., Jeribi, A., Ghorbel, A. \& Hachicha, N. 2021. Hedging
stock market prices with WTI, gold, VIX and cryptocurrencies: A
comparison between DCC, ADCC and GO-GARCH models. \emph{International
Journal of Emerging Markets}. (ahead-of-print).

\leavevmode\vadjust pre{\hypertarget{ref-ho2004analysis}{}}%
Ho, K.-Y. \& Tsui, A.K. 2004. An analysis of the sectoral indices of
tokyo stock exchange: A multivariate GARCH approach with time varying
correlations. In \emph{Stochastic finance, autumn school and
international conference}.

\leavevmode\vadjust pre{\hypertarget{ref-jse_website}{}}%
JSE. 2024. \emph{The all share index (J203)}. {[}Online{]}, Available:
\url{https://www.jse.co.za/all-share-j203}.

\leavevmode\vadjust pre{\hypertarget{ref-katzke2013south}{}}%
Katzke, N. 2013. South african sector return correlations: Using DCC and
ADCC multivariate GARCH techniques to uncover the underlying dynamics.
\emph{Stellenbosch Economic Working Papers: 17/13}. 1--31.

\leavevmode\vadjust pre{\hypertarget{ref-tsay2013}{}}%
Tsay, R.S. 2013. \emph{Multivariate time series analysis: With r and
financial applications}. John Wiley \& Sons.

\end{CSLReferences}

\newpage

\hypertarget{appendix}{%
\section{Appendix}\label{appendix}}

\begin{figure}[H]

{\centering \includegraphics{WriteUp_files/figure-latex/FigureA1-1} 

}

\caption{Return Persistence: Financials \label{FigureA1}}\label{fig:FigureA1}
\end{figure}

\begin{figure}[H]

{\centering \includegraphics{WriteUp_files/figure-latex/FigureA2-1} 

}

\caption{Return Persistence: Resources \label{FigureA2}}\label{fig:FigureA2}
\end{figure}

\begin{figure}[H]

{\centering \includegraphics{WriteUp_files/figure-latex/FigureA3-1} 

}

\caption{Return Persistence: Industrials \label{FigureA3}}\label{fig:FigureA3}
\end{figure}

\begin{figure}[H]

{\centering \includegraphics{WriteUp_files/figure-latex/FigureA4-1} 

}

\caption{Autocorrelation Functions: Financials \label{FigureA4}}\label{fig:FigureA4}
\end{figure}

\begin{figure}[H]

{\centering \includegraphics{WriteUp_files/figure-latex/FigureA5-1} 

}

\caption{Autocorrelation Functions: Resources \label{FigureA5}}\label{fig:FigureA5}
\end{figure}

\begin{figure}[H]

{\centering \includegraphics{WriteUp_files/figure-latex/FigureA6-1} 

}

\caption{Autocorrelation Functions: Industrials \label{FigureA6}}\label{fig:FigureA6}
\end{figure}

\bibliography{Tex/ref}





\end{document}
